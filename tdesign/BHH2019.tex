% Created 2020-08-27 Thu 01:02
% Intended LaTeX compiler: pdflatex
\documentclass[11pt]{article}
\usepackage[utf8]{inputenc}
\usepackage[T1]{fontenc}
\usepackage{graphicx}
\usepackage{grffile}
\usepackage{longtable}
\usepackage{wrapfig}
\usepackage{rotating}
\usepackage[normalem]{ulem}
\usepackage{amsmath}
\usepackage{textcomp}
\usepackage{amssymb}
\usepackage{capt-of}
\usepackage{hyperref}
\author{Edward Kim}
\date{\today}
\title{BHH2019}
\hypersetup{
 pdfauthor={Edward Kim},
 pdftitle={BHH2019},
 pdfkeywords={},
 pdfsubject={},
 pdfcreator={Emacs 26.3 (Org mode 9.4)}, 
 pdflang={English}}
\begin{document}

\maketitle
\tableofcontents


\section{Major Findings to Investigate/Understand}
\label{sec:org9e50a7b}
\subsection{Random Haar Unitary requires exponential number of two-qubit gates and random bits.}
\label{sec:org82a1496}
This makes it infeasible for practice to select such a unitary according to Haar measure.
\href{https://arxiv.org/pdf/quant-ph/9508006.pdf}{Knill. Approximations by Quantum Circuits}

\subsection{Best result for a some time was that polynomial random quantum circuits are approximate 2-designs}
\label{sec:orge6ed15b}
\subsection{A unitary chosen under Haar measure can be thought of sampling from a uniform distribution over all unitaries in the unitary group.}
\label{sec:org549b286}
A quantum expander's rapid mixing time (polynomial in \(t\) for a \$t\$-design) means that we can rapidly converge to such a unitary.

We can then think about a local random circuit on n qudits as a random walk such that we pick some index i and some unitary \(U_{i,i+1} \in \mathbb{U}(d^2)\) and we apply \(U_{i.i+1}\) on the qubits \(i\) and \(i+1\). This gives us a distribution over \(\mathbb{U}(d^n)\)

\subsection{Folklore result: An overwhelming number of pure quantum states on n qubits is indistinguihable from the maximally mixed state if restricted to measurements implemented with subexponential-sized quantum circuits.}
\label{sec:orga3a6da1}
\href{https://arxiv.org/pdf/0812.3001.pdf}{Are random pure states useful for quantum computation?}

\subsection{The proof of the main result concerns bounding the mixing time by the spectral gap (Cheeger-like inequality)}
\label{sec:org554aa59}
\href{https://arxiv.org/pdf/0910.0913.pdf}{Convergence rates for arbitrary statistical moments of random quantum circuits
}
\subsection{Previous papers have weaker results and are probably worth taking a look for learning techniques behind proofs.}
\label{sec:org16d9be3}
\href{https://arxiv.org/pdf/0811.2597.pdf}{Efficient Quantum Tensor Product Expanders and k-designs}
\href{https://link.springer.com/content/pdf/10.1007/s00220-009-0873-6.pdf}{Random Quantum Circuits are Approximate 2-designs}

\section{Questions to consider.}
\label{sec:org80761cf}

\subsection{Is it worth learning the tensor-product expander constructions now? Maybe I should make a short article about it?}
\label{sec:orgf9d2e3d}
\end{document}
